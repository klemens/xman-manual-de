\documentclass[a4paper,11pt]{article}
\usepackage{a4wide,ngerman,url,graphicx}
\usepackage[utf8]{inputenc}
\usepackage{hyperref}
\hypersetup{
    colorlinks,
    citecolor=black,
    filecolor=black,
    linkcolor=black,
    urlcolor=black
}
\setcounter{secnumdepth}{2}

\parindent0pt
\parskip4pt

\newcommand{\bild}[2]{
    \begin{center}\includegraphics[width=#1\textwidth]{#2}\end{center}
}

\newcommand{\knopf}[1]{{\sc #1}}

\author{Daniel Gerber, Hans-Gert Gräbe\\ 
Klemens Schölhorn, Oliver Swoboda, Robert Seidler\\
Institut für Informatik, Universität Leipzig}

\title{Die OpenOLAT-Erweiterung Xman -- Handbuch}
\date{Überarbeitete Version -- 30.05.2013}
\begin{document}
\maketitle

\vspace{3em}

\tableofcontents

\clearpage
\section*{Einführung}

\subsection*{Über dieses Handbuch}

Dieses Handbuch soll Sie in die Arbeit mit der Prüfungsverwaltungssoftware
Xman einführen und Ihnen Konzeption sowie grundlegende Funktionalität der
Software nahe bringen. Dieses Handbuch ist für Nutzer geschrieben, die noch
keine Erfahrung mit Xman haben. Ihnen wird Schritt für Schritt erklärt, wie
einzelne Funktionen zu verwenden sind.  Fortgeschrittene Nutzer können dieses
Handbuch als Referenz verwenden und sich am Inhaltsverzeichnis orientieren, um
eine bestehende Frage möglichst schnell beantwortet zu bekommen.

Sollten in diesem Dokument Fehler enthalten sein oder ein Sachverhalt nicht
detailliert genug beschrieben sein, bitten wir Sie, uns
Verbesserungsvorschläge über die Emailadresse
\begin{center}
  \url{olat-portal@informatik.uni-leipzig.de} 
\end{center}
zu unterbreiten, um die Qualität dieses Dokuments durch Ihre Mithilfe weiter
zu verbessern.

\subsection*{Aufbau dieses Handbuchs}

Das Handbuch ist in drei Kapitel unterteilt. Im ersten Kapitel wird das Portal
{\em aus studentischer Sicht} beschrieben. Es werden
grundlegende Fragen wie die Registrierung an der Plattform und die Beantragung der
Studentenakte sowie die Einschreibung zu Prüfungen beschrieben.

Das zweite Kapitel widmet sich der Darstellung der Portalfunktionen {\em für
das Studienbüro}. Es wird hier darauf eingegangen,
wie Studiengänge und Studentenakten verwaltet werden.

Das letzte Kapitel des Handbuchs stellt die von {\em Prüfern}
nutzbare Funktionalität zur Erstellung und Verwaltung von Prüfungen vor.
Hierzu wird schrittweise erklärt, wie eine Prüfung erstellt und bearbeitet
werden kann.

\clearpage
\section{Hilfe für Studierende}

\subsection{Registrierung und Anmeldung am OpenOLAT-PA-Portal}

Am OpenOLAT-PA-Portal erfolgt die Authentifizierung für Studierende über den
zentralen Studserv-Account der Universität mit den Ihnen im Zuge der
Immatrikulation bekanntgegebenen Zugangsdaten.  Während des ersten Zugangs zum
OpenOLAT-PA-Portal werden einige Daten abgefragt, die für das Prüfungsmanagement
erforderlich sind, aber vom Studserv-Account nicht mit übertragen werden.
Geben Sie diese persönlichen Daten entsprechend des vorgegebenen Formulars ein
und bestätigen Sie die Eingabe.

\subsection{Studentenakte}

Im zweiten Schritt müssen Sie sich eine {\em Studentenakte}
erstellen, in der die relevanten operativen Informationen zum
Prüfungsgeschehen gespeichert sind. 

Dazu klicken Sie unter \knopf{Home $>$ Meine Studentenakte} auf den Button
\knopf{Erstellen}.

\bild{1}{ESA-Erstellen}

Füllen Sie die noch offenen Felder des erscheinenden Formulars aus und
kontrollieren Sie die Richtigkeit der Angaben.

\bild{1}{ESA-Eingaben}

Als {\em Login} und {\em E-Mail-Adresse} werden Ihre Studserv-Daten verwendet.
Diese können nicht geändert werden. Beachten Sie, dass Ihre {\em Matrikelnummer}
ausschließlich aus Zahlen bestehen darf und nach erstmaliger Eingabe ebenfalls
nicht mehr geändert werden kann. Wählen Sie Ihren {\em Studiengang}
aus der Liste der vorgegebenen Einträge.  Falls Ihr Studiengang dort nicht
angegeben ist, wählen Sie den Studiengang {\em Sonstiges}.

Haben Sie die Studentenakte erfolgreich beantragt, wird diese auf der gleichen
Seite direkt angezeigt.

Sie können Ihren Namen und Studiengang jederzeit ändern, z.\,B. beim
Übergang vom Bacherlor- zum Masterstudiengang, einer Heirat oder ähnlichem.

Sie können Ihre eigene Akte jederzeit über \knopf{Home $>$ Meine Studentenakte}
einsehen. Beachten Sie, dass das Portal ausschließlich der operativen
Unterstützung des Prüfungsgeschehens dient und nur die endgültigen Ergebnisse
Eingang in Ihre zentral an der Universität geführten Unterlagen finden. Bei
Unstimmigkeiten zum Inhalt der Studentenakte nehmen Sie bitte Kontakt zum Studienbüro
auf.

\subsection{Einschreibung zu einer Prüfung}

Zu einer Prüfung können Sie sich nur einschreiben, wenn Sie sich bereits eine
{\bf Studentenakte} angelegt haben.

Nachdem Sie im Reiter \knopf{Lernressourcen} die gewünschte Prüfung über den
Katalog, die Suchmaske oder über die Liste aller Prüfungen gefunden und
gestartet haben, erhalten sie eine Übersicht folgender Art:

\bild{1}{Pruefung-Einschreiben}

Handelt es sich bei der gewählten Prüfung um eine mündliche Prüfung, wird
Ihnen -- wie im Bild dargestellt -- eine Liste mit noch offenen Terminen
angeboten.  Dies gilt auch für schriftliche Prüfungen, allerdings gibt es hier
lediglich einen Termin zur Auswahl. Um sich zu einer Prüfung einzuschreiben,
wählen Sie einen für Sie passenden freien Termin durch Drücken des
\knopf{Einschreiben}-Buttons. Nach diesem Schritt zeigt Ihnen das System über die
Spalte \knopf{Status} an, ob Sie erfolgreich registriert wurden. Zusätzlich erhalten Sie
zur Bestätigung eine Email vom System mit allen relevanten Daten. Außerdem kann der
weitere Prüfungsverlauf nun in der eigenen Studentenakte verfolgt werden.
 
\subsection{Prüfungsergebnisse in der Studentenakte einsehen}

Nachdem die Bewertung Ihrer Prüfungsleistung eingetragen wurde
erhalten Sie eine Benachrichtigungsmail. Ab diesem Zeitpunkt können Sie
das Ergebnis Ihrer Prüfung in Ihrer Studentenakte einsehen.  Ihre Studentenakte hat etwa
das folgende Aussehen:

\bild{1}{ESA-Ansicht}

\clearpage
\section{Hilfe für das Studienbüro}

\subsection{Registrierung am OpenOLAT-PA-Portal}

Für die Registrierung am Portal wenden Sie sich bitte an die
Portaladministration. Diese legt für Sie einen neuen Account an, schaltet die
erforderlichen Rechte frei und trägt Sie außerdem in die Arbeitsgruppe {\em
  Prüfer} ein.

Über den Reiter \knopf{Gruppen} oder über das Portlet \knopf{Home $>$ Home $>$ Meine
  Gruppen} können Sie zum Arbeitsbereich dieser Gruppe wechseln, wo
insbesondere im {\em Forum} und {\em Wiki} aktuelle Informationen zum Betrieb
des Portals aus Sicht der Prüfer und des Studienbüros zu finden sind.

Über \knopf{Home $>$ Einstellungen} können Sie das Ihnen zugewiesene
provisorische Passwort ändern.

\subsection{Prüfungsergebnisse exportieren}

Um die Ergebnisse einer Prüfung zu exportieren klicken Sie in der
Prüfungsansicht auf das Symbol \knopf{Tabelle herunterladen}, im folgenden
Bild gelb markiert.

\bild{1}{Ergebnis-Exportieren}

OpenOLAT generiert nun eine xls-Datei mit den in der Tabelle vorhandenen Daten und
bietet diese zum Herunterladen für die Weiterverarbeitung an.

Es besteht die Möglichkeit, die Tabelle vorher anzupassen. Durch Klicken auf
den im Bild rot markierten Bereich \knopf{Tabelle anpassen} ist es möglich, in
der Tabelle dauerhaft Spalten hinzuzufügen oder zu entfernen. Entfernte Spalten werden
auch beim Export nicht berücksichtigt.

\subsection{Studentenakten einsehen}

Im Reiter \knopf{Studienbüro} unter dem Menüpunkt \knopf{Studentenakten}
können sie alle vorhandenen Studentenakten einsehen und über
\knopf{Suchen} auch nach einer spezifischen Akte suchen.

\subsection{Studentenakte bearbeiten}

Um eine bestimmte Studentenakte zu bearbeiten, klicken Sie auf
die in der Tabelle angezeigte Matrikelnummer des jeweiligen Studierenden.
Es öffnet sich ein neuer Reiter mit der gewählten Studentenakte:

\bild{1}{ESA-Edit}

Sie können hier einen \knopf{Kommentar hinzufügen}.

Sollten Sie persönliche Informationen wie Vorname, Nachname,
Matrikelnummer, E-Mail-Adresse oder Studiengang ändern wollen, verwenden Sie
bitte die vorgegebenen Werkzeuge unter dem Reiter \knopf{Benutzerverwaltung}
oder wenden Sie sich an den Administrator des Systems. Wichtig ist hier, dass
Ihnen am Portal die Rolle {\em Benutzerverwalter} vom Administrator zugewiesen wurde.

\clearpage
\section{Hilfe für Prüfende}

\subsection{Registrierung am OpenOLAT-PA-Portal}

Die Registrierung für Prüfende verläuft auf dieselbe Weise wie im letzten
Kapitel für die Rolle Studienbüro beschrieben.

\subsection{Prüfung erstellen}

Um eine Prüfung zu erzeugen, verwenden sie das Werkzeug \knopf{Herstellen $>$
Prüfung} im Reiter \knopf{Prüfungen}. Nach Klicken auf dieses Werkzeug
folgt der erste Schritt der Erstellung einer Prüfung mit der Auswahl des
Prüfungstyps. Diese Einstellung kann später nicht mehr geändert werden.

\bild{1}{Pruefung-Erzeugen-Typ}

Für den Prüfungstyp {\em mündlich} können Sie im späteren Verlauf beliebig
viele Termine zur Prüfung hinzufügen. Zu jedem solchen Termin kann sich
jeweils genau ein Student einschreiben. Für den Prüfungstyp {\em schriftlich}
wird später lediglich ein Termin erzeugt.

Im weiteren Verlauf werden sie aufgefordert, den Namen der Prüfung sowie eine kurze
Beschreibung einzugeben. Sie können die Beschreibung hier sehr kurz halten, da
eine detailliertere Beschreibung der Prüfung in einem späteren Schritt
abgefragt wird. Halten sie sich aber bei dem Namen der Prüfung an die vom
Studienbüro vorgegebene Namenskonvention. 

Starten Sie im nächsten Schritt den Editor, um die Prüfung genauer zu
definieren.

\bild{1}{Pruefung-Erzeugen-Editor}

Der Prüfungseditor öffnet sich wie oben beschrieben in einem neuen Reiter mit
dem Prüfungsnamen als Titel. Zusätzlich wird Ihnen der Titel mit dem Prüfungstyp
({\em mündlich/schriftlich}) in der {\em Brotkrumennavigation} (rote Markierung)
angezeigt. Sie können diese Navigation jederzeit verwenden, um die Prüfung
in der späteren Ansicht zu sehen. 

Im ersten Reiter werden die Prüfungskommentare wie mitzubringende
Unterlagen, vorzubereitende Prüfungsthemen oder Details zum Modul und Prüfer eintragen.
Um die Änderungen
dauerhaft zu machen, bestätigen Sie bitte mit \knopf{Speichern}. Unter
\knopf{Einschreibung} können sie den Anmelde- und Abmeldezeitraum
festlegen. Hier ist darauf zu achten, dass der Beginn des Anmeldezeitraums
zeitlich vor den beiden Terminen für An- und Abmeldeschluss liegt.

Nach der Bestätigung mit \knopf{Speichern} erscheint ein neuer Reiter
\knopf{Termin(e)}. Hier wird der Prüfung der eigentliche Termin zugeordnet.
Die beiden Prüfungstypen {\em mündlich/schriftlich} teilen sich die Attribute
Datum, welches über einen Datumsauswahldialog gewählt wird, Ort und die Dauer
in Minuten der Prüfung. Sollten Sie den Prüfungstyp {\em mündlich} gewählt
haben, stehen Ihnen noch zwei weitere Attribute zur Verfügung. Das System
erzeugt dann die bei {\em Anzahl} angegebene Zahl von konsekutiven Terminen
entsprechend den Angaben bei {\em Pause} und {\em Dauer}.  Sie können auf
diese Weise mehrfach Termine erzeugen und später auch erzeugte Termine wieder
löschen.  Bei der Wahl der Termine sollten Sie vorab mglw.\ Rücksprache mit
dem Studienbüro halten, um eventuelle terminliche Kollisionen mit anderen
Prüfungen zu vermeiden.

Unter dem Reiter \knopf{Vormerken} können Sie einstellen, ob Studierende mit
Ihrer Anmeldung für die Prüfung nur vorgemerkt oder bereits verbindlich
registriert werden.  Der Vormerkemodus ist besonders für die Fälle vorgesehen,
wenn vorab spezielle Zulassungsvoraussetzungen zu prüfen sind oder Sie nach
Anmeldeschluss Lücken in Ihrem Prüfungsplan durch zeitliche Verschiebung von
Prüfungen zu schließen beabsichtigen.  

Im Vormerkemodus bekommen die Studentinnen und Studenten zunächst den Status {\em Vorgemerkt}. Sie entscheiden dann später für jeden Studierenden einzeln,
ob dieser den Status {\em Registriert} bekommt und somit zu den zugelassenen Prüflingen gehört.
Öffnen Sie dazu den Reiter dieser Prüfung, wählen Sie die zu registrierenden
Studentinnen und Studenten über das Wahl-Kästchen aus und führen
Sie die Aktion \knopf{Registrieren} aus. Die Registrierung kann über die Aktion \knopf{Vormerken} wieder rückgängig gemacht werden.

Unter dem Reiter \knopf{Mehrfacheinschreibung} können Sie einstellen, ob
Studierende sich zu mehreren Prüfungsterminen anmelden können.
Die {\em Mehrfacheinschreibung} ist nur für {\em mündliche} Prüfungen möglich.

Im Reiter \knopf{Katalog} können Sie Ihre Prüfung in den Prüfungskatalog
eintragen und damit einem bestimmten Prüfungszeitraum zuordnen. Sollte der
gewünschte Zeitraum nicht im Katalog vorhanden sein, wenden sie sich bitte an
das Studienbüro.

Nachdem die Prüfung vollständig angelegt wurde, {\bf muss die Sichtbarkeit der
  Prüfung noch geändert werden}. Starten Sie dafür die \knopf{Detailansicht}
der Prüfung und wählen Sie das am rechten Bildschirmrand erscheinende Werkzeug
\knopf{Einstellungen ändern} aus. Unter \knopf{Zugriff haben:} müssen Sie die
Prüfung für {\em Alle registrierten OpenOLAT-Benutzer} freigeben.  

Ihre Prüfung steht jetzt den Prüflingen zur Einschreibung zur Verfügung.

\subsection{Prüfung abschließen}

Um eine Prüfung abzuschließen, wählen Sie die Aktion \knopf{Prüfung abschließen}
im Prüfungseditor. Nach dem Abschließen einer Prüfung sind keine Änderungen an
dieser mehr möglich. Diese Funktion können Sie nutzen, wenn Ihre Prüfung beendet
und alle Ergebnisse eingetragen wurden.

\subsection{Kontakt mit Prüfling aufnehmen}

Um Kontakt mit einem Prüfling aufzunehmen, wählen Sie in der Liste der
eingetragenen Studentinnen und Studenten über das Wahl-Kästchen
diejenigen aus, denen Sie eine E-Mail senden möchten und führen Sie die Aktion
\knopf{E-Mail senden} aus.

\subsection{Noten und Kommentare eintragen}

Um die Noten der Prüflinge einzutragen, wählen sie in der Liste der Prüflinge
die entsprechenden Studenten mit einem
Klick auf das Wahl-Kästchen aus und führen die Aktion \knopf{Noten
  eintragen} aus. Alle in einem Vorgang ausgewählten Studierenden erhalten die
gleiche Note.

Um zu Prüflingen einen Kommentar einzutragen, wählen Sie diese
wie gewohnt über das Wahl-Kästchen aus und führen die Aktion \knopf{Kommentare
 eintragen} aus. Ist bei einem Prüfling schon ein Kommentar vorhanden,
wird dieser durch den neuen Kommentar ersetzt. Bei Auswahl von nur einem
Prüfling kann der vorhandene Kommentar editiert werden.

\subsection{Prüflinge nachtragen und entfernen}

Falls eine am Portal registrierte Studentin oder ein Student den festgelegten Anmeldeschluss
verpasst hat oder aus anderen Gründen für die Prüfung nicht registriert ist,
können diese (auch nach Anmeldeschluss und sogar nach Ablauf der Prüfung)
manuell nachgetragen werden. Wählen Sie dazu in der Prüfungsansicht \knopf{Nachtragen}
(nach Auswahl des gewünschten Termins) bzw. \knopf{Teilnehmer nachtragen}
aus. Es öffnet sich ein Nutzersuchformular, in dem Sie nach dem
nachzutragenden Prüfling suchen können. Falls der Prüfling
noch keine Studentenakte angelegt hat, wird eine entsprechende
Fehlermeldung angezeigt.

Um einen bereits {\em registrierten} oder {\em vorgemerkten} Prüfling abzumelden, wählen
Sie ihn aus und führen die Aktion \knopf{Austragen} aus.

\end{document}
